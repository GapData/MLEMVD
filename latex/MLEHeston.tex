\documentclass{article}

% The following \documentclass options may be useful:

% preprint      Remove this option only once the paper is in final form.
% 10pt          To set in 10-point type instead of 9-point.
% 11pt          To set in 11-point type instead of 9-point.
% authoryear    To obtain author/year citation style instead of numeric.
\usepackage[hmargin=2cm,vmargin=1.5cm]{geometry}
\usepackage{amsmath}
\usepackage{epsfig,times}
%\usepackage{algorithm}
%\usepackage{algorithmicx,algpseudocode}
\usepackage[linesnumbered, ruled]{algorithm2e}
\SetKwRepeat{Do}{do}{while}%
%\usepackage[ruled]{algorithm2e}
%\usepackage{algorithmic,algorithm2e,float}
\usepackage{float}
\usepackage{listings}
\usepackage{color}
\usepackage{graphicx}

\usepackage{listings}
\usepackage{color}

%
% By default LaTeX is quite picky with float placement,
% relax a bit in order to keep this document within 2 pages
%
\renewcommand{\topfraction}{0.95}
\renewcommand{\bottomfraction}{0.95}
\renewcommand{\textfraction}{0.05}
\renewcommand{\floatpagefraction}{0.35}

\lstnewenvironment{code}[1][]%
  {\minipage{\linewidth} 
   \lstset{basicstyle=\ttfamily\footnotesize,frame=single,#1}}
  {\endminipage}

\definecolor{theWhite}{gray}{0.9}
\definecolor{theBlack}{gray}{0.0}
\newcommand{\be}{\begin{equation}}
\newcommand{\ee}{\end{equation}}
\newcommand{\bi}{\begin{itemize}}
\newcommand{\ei}{\end{itemize}}
\newcommand{\eps}{\epsilon}

\begin{document}

\special{papersize=8.5in,11in}
\setlength{\pdfpageheight}{\paperheight}
\setlength{\pdfpagewidth}{\paperwidth}

\title{Calibration of the Heston model using Maximum Likelihood Estimation}
%\subtitle{Building Scalable Analytics Applications}

\author{Matthew Dixon\\
           Stuart School of Business\\
	   Illinois Institute of Technology\\
           mdixon7@stuart.iit.edu}


\maketitle

This document briefly outlines the Heston model calibration approach implemented in the R package\\

 \verb|https://github.com/mfrdixon/MLEMVD|. 

\section{Heston Model} \label{sect:heston}
Under the pricing measure $Q$, the Heston model describes the evolution of the log of stock price $s_t =ln~S_t$ whose variance $Y_t$ is given by a mean reverting square root process:
\begin{eqnarray}
ds_t &=& (a + bY_t)dt  + \sqrt{Y_t}dW_1{Q}(t) ,\\
dY_t &=& \kappa'(\theta' - Y_t)dt  + \sigma \sqrt{Y_t}dW_2^{Q}(t),
\end{eqnarray}
where
\be
a=r-d, \qquad b= -\frac{1}{2},
\ee
A key characteristic of the model is that the Wiener processes are correlated $dW^Q_1\cdot dW_2^Q=\rho dt$. This feature enables the model to exhibit the 'leverage effect'.

To find the model parameters from option prices first requires adjustment of the model to account for market risk, so that under the objective pricing measure $P$
\begin{eqnarray}
ds_t &=& (a + bY_t)dt  + \sqrt{Y_t}dW_1{Q}(t) ,\\
dY_t &=& \kappa(\theta- Y_t)dt  + \sigma \sqrt{Y_t}dW_2^{Q}(t),
\end{eqnarray}
where
\be
a=r-d, \qquad b=\lambda_1(1-\rho^2) + \lambda_2\rho -\frac{1}{2}, \qquad \kappa=\kappa'-\lambda_2\sigma, \qquad \theta=\left(\frac{\kappa +\lambda_2\sigma}{\kappa}\right)\theta'.
\ee

The parameter set $\mathbf{p}:=[\kappa, \theta, \sigma, \rho, \lambda_1, \lambda_2]$ and the additional non-linear constraint (the Feller condition) $2\kappa\theta - \sigma^2>0$ is imposed during the calibration to ensure that $Y_t$ is positive.



\section{Likelihood function estimation}
Given a set of observed underlying and ATM constant maturity option prices $g_t:=[S_t; C_t]$ sampled at dates $t_0,t_1,\dots, t_n$, the likelihood function takes the form:
\be
l_n(\mathbf{p}):=\frac{1}{n}\sum_{i=1}^nl_G(\Delta t_i,g(t_i)|g(t_{i-1}));\mathbf{p})
\ee
where
\be
l_G(\Delta, g|g_0;\mathbf{p}) := ln f_G(\Delta,g|g_0;\mathbf{p}) = - ln J_t(\Delta, g|g_0;\mathbf{p}) + l_X(\Delta, f^{-1}(g;\mathbf{p})| f^{-1}(g_0;\mathbf{p});\mathbf{p})
\label{eq:likelihood}
\ee
and $l_X$ denotes the likelihood function of the partially observed state vector $x_t:=[ln~S_t, Y_t]$ evaluated at each date $t_0,t_1,\dots, t_n$. Here $\Delta t_i:=t_i-t_{i-1}$ denotes the time step between observations. $J_t$ denotes the Jacobian of the option price with respect to $Y_t$, which is equivalently to vega.

\section{Calibration}

Given a sequence of observed underlying and corresponding near expiry, constant maturity, ATM options, we follow the these steps

\begin{itemize}
\item Initialize the unknown parameter vector to the model
\item For each new parameter set $\mathbf{p}$ generated by the numerical optimization routine, compute the value of $Y_t$ which satisfies the option price. Note this requires solving a nested one dimensional convex optimization, with linear bound constraints, so that $C_t \rightarrow \hat{Y}_t$.
\item With $\hat{Y}_t$ and $\mathbf{p}$ compute $\nu$ of the option and thus the Jacobian term in Equation \ref{eq:likelihood}.
\item Using $\hat{x}_t=[ln~S_t, \hat{Y}_t]$ and $\mathbf{p}$ compute $l_X(\Delta, f^{-1}(g;\mathbf{p})| f^{-1}(g_0;\mathbf{p});\mathbf{p})$
\item Minimize the log likelihood $l_G$ over the parameters subject to the Feller condition. 
\end{itemize}


\end{document}